\chapter{OMNeT++编程接口}

\begin{summary}
在完成第五章后,考虑需要在之前加一章节关于OMNeT++类说明,在这个仿真软件中,主要使用的语言是C++,因此大多数数据类型是类或者结构,本章还是走其他技术书一样的老路线,注释这些数据类型,对类成员函数进行说明,可能与第五章有些重复的地方,但是其五章更多的偏向于实际应用,可能读者看过这里后,会发现OMNeT++接口是真好用。\\ \\
\end{summary}

\section{循规蹈矩}


\section{类说明}



\subsection{cModule}
为了能更好的解释这个的库的使用,程序清单4.1为类cModule原型,cModule类在OMNeT++中表示一个节点的对象,这个节点可以是复合节点或者简单节点,通过这个类,程序员可以访问描述这个节点的.ned文件中设置的参数,或者是由omnetpp.ini传入的参数。简而言之,我们最后就是面向这些类进行网络设计。\\


\subsection{cPar}


\subsection{cGate}


\subsection{cTopology}


\subsection{cExpression}



