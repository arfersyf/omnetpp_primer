\part{Part One}
\chapter{OMNeT++}

\begin{summary}
OMNeT++是一个网络仿真工具,支持以太网、无线网等协议仿真,同时提供友好的仿真界面以及3D显示。\\ \\
\end{summary}

\section{OMNeT++简介}

OMNeT++,一个基于eclipse开发套件的开源网络仿真工具,目前主要在高校实验室进行一些网络仿真测试,对一些算法进行对比,它可以供使用者进行完成以下开发:
\begin{itemize}
\item C/C++开发;
\item 网络仿真程序设计。
\end{itemize}

毫无疑问,基于eclipse的开发工具肯定能支持普通的C/C++工程。
另外,在OMNeT++上网络仿真设计领域的优势在于,它是一个开源的项目,对大量的网络模型都提供代码支持。但是问题在于国内的确没有什么社区支持,出现问题只能自己解决,其实对于开源的项目大多存在这种问题,往往开源的项目,使用起来难度较大,开源项目往往比那些商业的软件开发难度较大,支持也较少,开源可不代表简单。

OMNeT++对初学者能力要求高,它假定使用者对编程有一定了解的,对eclipse开发环境也是特别熟悉的,另外这是一个网络仿真的软件,需要你对计算机网络有足够的认识,它提供了大量现有各种网络的仿真例子,如果你对网络认识足够强,那么这个软件你用起来会感到特别顺手。

目前有大量的开源仿真库用于OMNeT++环境,拥有丰富的外文资料,官方将其分为两类,包括Supported Models和Contributed Models:
\begin{itemize}
	\item Supported Models \\
	模型库的开发处于激活状态,有开发者在维护,定期会推出新的版本。
	\item Contributed Models \\
	完成后只推出过一次或几次版本,目前没有人在维护。\\ \\
\end{itemize}


\section{OMNeT++开源库}

下面简单介绍一下几种常见的开源库。


\begin{itemize}
\item INET

由 Simucraft 公司主持开发,用于仿真有线及无线网络。

应用层协议:

- HTTP、FTP、Telnet、不同优先级的 Video、Ping

传输层协议:

- TCP、UDP、RTP ( RealtimeTransport Protocol )

网络层协议:

- IPv4、IPv6、ICMP、ARP、MPLS、LDP、RSVP、OSPF、Mobile IPV6、AODV、DSDV、DSR

数据链路层协议:

- Ethernet、PPP、IEEE 802.11、FDDI、Token Ring

- 官网:http://inet.omnetpp.org

\item INETMANET

由 Simucraft 公司主持开发,用于仿真无线、有线网络,在INET 的基础上增加了大量的 MANET 协议,INETMANET= INET+MANET,在INET的基础上增加:

- 802.11a,g:Ieee80211aMac, Ieee80211gMac, Ieee80211aRadioModel, Ieee80211gRadioModel

- Ieee80211Mesh,Ieee80211MeshMgmt

- radiomodels: TwoRayModel, ShadowingModel, qamMode

- Ns2MotionMobility

- ARP:global ARP cache

- AODV,DSDV, DSR, DYMO, OLSR

- 官网:http://inet.omnetpp.org

\item Mobility Framework

- 由 Simucraft 公司主持开发

- 是一个无线传感器仿真模型库

- 绝大多数协议已经被 INET 吸收

- 官网:http://mobility-fw.sourceforge.net/hp/index.html

\item SensorSimulator

- 美国路易斯安娜州立大学开发

- 用于仿真无线传感器网络

- 官网:\url{http://csc.lsu.edu/sensor_web/}

\item Castalia

- 澳大利亚国家信息技术中心(NICTA)开发

- 是一个基于 OMNeT++ 的侧重于无线网络的仿真器

- 基于实测数据的高级 channel/radio 模型

- Radio 详细的状态转移,允许多传输功率电平

- 高度灵活的 physical process model

- 感应设备的噪声、偏差(bias)和功耗

- 节点时钟漂移,CPU 功耗

- 资源监控,如超出功率限制(如 CPU 或内存)

- 拥有大量可调参数的mac协议

- 用于设计优化和扩展

- 官网:https://github.com/boulis/Castalia


\item OverSim

- 德国德国卡尔斯鲁厄大学开发

- 用于仿真点对点(p-to-p)协议,如chard,GIA等

- 官网:\url{http://www.oversim.org} \\ \\

\end{itemize}





\section{目录}

本手册与现有的那两本书风格不同,我希望读者通过此手册可以快速的上手OMNeT++,快速的掌握OMNeT++提供的各种接口,目前包括以下内容:

\begin{itemize}
	\item OMNeT++的安装
	\item INET库的安装 INET库的基本使用
	\item OMNeT++个性化设置
	\item OMNeT++工程设计技巧
	\item cModule | cPar | cGate | cTopology相关类使用
	\item 仿真结果分析
	\item 仿真错误记录
\end{itemize}

%---------------------------------------------------------------------------


